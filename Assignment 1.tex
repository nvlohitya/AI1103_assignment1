\documentclass[12pt]{journal}
\author{Nanduri Venkat Lohitya-MS20BTECH11015}
\twocolumn

\usepackage{setspace}
\usepackage{gensymb}
\singlespacing
\usepackage[cmex10]{amsmath}

\usepackage{amsthm}

\usepackage{mathrsfs}
\usepackage{txfonts}
\usepackage{stfloats}
\usepackage{bm}
\usepackage{cite}
\usepackage{cases}
\usepackage{subfig}

\usepackage{longtable}
\usepackage{multirow}

\usepackage{enumitem}
\usepackage{mathtools}
\usepackage{steinmetz}
\usepackage{tikz}
\usepackage{circuitikz}
\usepackage{verbatim}
\usepackage{tfrupee}
\usepackage[breaklinks=true]{hyperref}
\usepackage{graphicx}
\usepackage{tkz-euclide}


\usepackage{listings}
    \usepackage{color}                                            %%
    \usepackage{array}                                            %%
    \usepackage{longtable}                                        %%
    \usepackage{calc}                                             %%
    \usepackage{multirow}                                         %%
    \usepackage{hhline}                                           %%
    \usepackage{ifthen}                                           %%
    \usepackage{lscape}     
\usepackage{multicol}
\usepackage{chngcntr}

\begin{document}
\title{AI1103 : Assignment 1}
\maketitle


\section{Problem}
A card from a a pack of 52 cards is lost.\\From the remaining cards of the pack,two cards are drawn and are found to be both diamonds.Find the probability of the lost card being a diamond.

\section{Solution}
Let A1 be the probability that a diamond card is lost.\\
Let A2 be the probability that a card other then diamond is lost.\\
Let A be the probability that two cards are drawn and both are found to be diamonds.\\


We know by Bayer's Formula:\\eq 2.1
$$ P\left(\frac{A1}{A}\right) = \frac{P(A1).P\left(\frac{A}{A1}\right)}{P(A)}$$\\

Also we know by the theorem of total probability that :\\eq 2.2

$$ P(A) = P(A1).P\left(\frac{A}{A1}\right) + P(A2).P\left(\frac{A}{A2}\right) $$

Now we know that:\\eq 2.3
$$ P(A1) = \left(\frac{1}{4}\right)$$ eq 2.4\\

$$ P(A2) = \left(\frac{3}{4}\right)\\$$ eq 2.5\\

$$P\left(\frac{A}{A1}\right) = \left(\frac{12C2}{51C2}\right)\\$$ eq 2.6\\
$$P\left(\frac{A}{A2}\right) = \left(\frac{13C2}{51C2}\right)\\$$ 

From eq 2.2,2.3,2.4,2.5,2.6 we get:\\ eq 2.7
$$P(A) =  \left(\frac{1}{4}\right).\left(\frac{12C2}{51C2}\right) +\left(\frac{3}{4}\right).\left(\frac{13C2}{51C2}\right) $$



Now by using Bayes theorem (eq 2.1) we get:\\
$$ P\left(\frac{A1}{A}\right) = \frac{\frac{1}{4}.\left(\frac{12C2}{51C2}\right)}{P(A)}$$\\

$$ P\left(\frac{A1}{A}\right) = \frac{\frac{1}{4}.\left(\frac{12C2}{51C2}\right)}{\left(\frac{1}{4}\right).\left(\frac{12C2}{51C2}\right) +\left(\frac{3}{4}\right).\left(\frac{13C2}{51C2}\right)} $$\\

Now after simplifying we get:\\\\
$$P\left(\frac{A1}{A}\right) = \left(\frac{12C2}{12C2+3.13C2}\right)$$\\

Hence after simplifying we get:\\Ans
 $$ P \left(\frac{A1}{A}\right) = \left(\frac{11}{50}\right) = 0.22$$\\

\end{document}
